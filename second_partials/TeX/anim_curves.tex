\documentclass[margin=0pt]{standalone}
\usepackage[romanfamily=casual,lucidasmallscale, nofontinfo]{lucimatx}
%\usepackage[romanfamily=bright-osf, stdmathdigits=true,lucidasmallscale, nofontinfo]{lucimatx}
\linespread{1.04}	%  this value suits  scale=0.9  or lucidasmallscale
\usepackage{tikz}
\usepackage[export]{animate}

\begin{document}
\begin{animateinline}[autoplay,loop]{2}%
    \multiframe{240}{i=0+1}{%
    \begin{tikzpicture}
        \draw[white] (-8.8,-4.95) rectangle (8.8,4.95);
        \begin{scope}[xshift=-4cm]
            \def\scale{3.5}
            \colorlet{darkgreen}{green!50!black};
            \draw[thin,->] (-4,0) -- (4,0)node[below]{$x$};
            \draw[thin,->] (0,-4) -- (0,4)node[left]{$y$};
            \foreach \i in {-4,...,-1}{
            \pgfmathsetmacro{\k}{.1*\i};
            \pgfmathsetmacro{\coefa}{(1-sqrt(1-4*(\k)^2))/(2*\k)};
            \pgfmathsetmacro{\color}{10*(5+\i)};
            \draw[thick,color=yellow!\color!darkgreen] plot[domain=-1:1,samples=40] ({\scale*\coefa*(\x)^2},\scale*\x);
            }
            \draw[thick,color=darkgreen] plot[domain=-1:1,samples=40] ({-\scale*(\x)^2},\scale*\x);
            \begin{scope}
                \clip (-\scale,-\scale) rectangle (\scale,\scale);
                \foreach \i in {-4,...,-1}{
                \pgfmathsetmacro{\k}{.1*\i};
                \pgfmathsetmacro{\coefa}{(1+sqrt(1-4*(\k)^2))/(2*\k)};
                \pgfmathsetmacro{\color}{10*(5+\i)};
                \pgfmathsetmacro{\dom}{1/sqrt(-\coefa)}
                \draw[thick,color=yellow!\color!darkgreen] plot[domain=-\dom:\dom,samples=40] ({\scale*\coefa*(\x)^2},\scale*\x);
                }
            \end{scope}
            \foreach \i in {1,...,4}{
            \pgfmathsetmacro{\k}{.1*\i};
            \pgfmathsetmacro{\coefa}{(1-sqrt(1-4*(\k)^2))/(2*\k)};
            \pgfmathsetmacro{\color}{10*(5+\i)};
            \draw[thick,color=yellow!\color!darkgreen] plot[domain=-1:1,samples=40] ({\scale*\coefa*(\x)^2},\scale*\x);
            }
            \draw[thick,color=yellow] plot[domain=-1:1,samples=40] ({\scale*(\x)^2},\scale*\x);
            \begin{scope}
                \clip (-\scale,-\scale) rectangle (\scale,\scale);
                \foreach \i in {1,...,4}{
                \pgfmathsetmacro{\k}{.1*\i};
                \pgfmathsetmacro{\coefa}{(1+sqrt(1-4*(\k)^2))/(2*\k)};
                \pgfmathsetmacro{\color}{10*(5+\i)};
                \pgfmathsetmacro{\dom}{1/sqrt(\coefa)}
                \draw[thick,color=yellow!\color!darkgreen] plot[domain=-\dom:\dom,samples=40] ({\scale*\coefa*(\x)^2},\scale*\x);
                }
            \end{scope}
            \draw[thick,color=yellow!50!darkgreen] (-\scale,0) -- (\scale,0);
            \draw[thick,color=yellow!50!darkgreen] (0,-\scale) -- (0,\scale);
            \begin{scope}
                \clip (-\scale,-\scale) rectangle (\scale,\scale);
                \pgfmathsetmacro{\k}{.3};
                \pgfmathsetmacro{\coefa}{(1+sqrt(1-4*(\k)^2))/(2*\k)};
                \pgfmathsetmacro{\dom}{1/sqrt(\coefa)}
                \draw[very thick,color=blue] plot[domain={\dom*(240-\i)/240}:\dom,samples=40] ({\scale*\coefa*(\x)^2},\scale*\x);
            \end{scope}
        \end{scope}
        \node[below] at (4,4) {%
        \begin{minipage}[t]{7cm}
            For example, for $k = 3$, as $y$ approaches $0$, the point $(ky^2,y)$
            approaches the origin following the level curve corresponding to the
            level 
            \[\frac{3}{3^2 + 1} = 0.3\]
        \end{minipage}%
        };
    \end{tikzpicture}%
    }%
\end{animateinline}%
\begin{animateinline}[autoplay,loop]{2}
    \multiframe{240}{i=0+1}{%
    \begin{tikzpicture}
        \draw[white] (-8.8,-4.95) rectangle (8.8,4.95);
        \begin{scope}[xshift=-4cm]
            \def\scale{3.5}
            \colorlet{darkgreen}{green!50!black};
            \draw[thin,->] (-4,0) -- (4,0)node[below]{$x$};
            \draw[thin,->] (0,-4) -- (0,4)node[left]{$y$};
            \foreach \i in {-4,...,-1}{
            \pgfmathsetmacro{\k}{.1*\i};
            \pgfmathsetmacro{\coefa}{(1-sqrt(1-4*(\k)^2))/(2*\k)};
            \pgfmathsetmacro{\color}{10*(5+\i)};
            \draw[thick,color=yellow!\color!darkgreen] plot[domain=-1:1,samples=40] ({\scale*\coefa*(\x)^2},\scale*\x);
            }
            \draw[thick,color=darkgreen] plot[domain=-1:1,samples=40] ({-\scale*(\x)^2},\scale*\x);
            \begin{scope}
                \clip (-\scale,-\scale) rectangle (\scale,\scale);
                \foreach \i in {-4,...,-1}{
                \pgfmathsetmacro{\k}{.1*\i};
                \pgfmathsetmacro{\coefa}{(1+sqrt(1-4*(\k)^2))/(2*\k)};
                \pgfmathsetmacro{\color}{10*(5+\i)};
                \pgfmathsetmacro{\dom}{1/sqrt(-\coefa)}
                \draw[thick,color=yellow!\color!darkgreen] plot[domain=-\dom:\dom,samples=40] ({\scale*\coefa*(\x)^2},\scale*\x);
                }
            \end{scope}
            \foreach \i in {1,...,4}{
            \pgfmathsetmacro{\k}{.1*\i};
            \pgfmathsetmacro{\coefa}{(1-sqrt(1-4*(\k)^2))/(2*\k)};
            \pgfmathsetmacro{\color}{10*(5+\i)};
            \draw[thick,color=yellow!\color!darkgreen] plot[domain=-1:1,samples=40] ({\scale*\coefa*(\x)^2},\scale*\x);
            }
            \draw[thick,color=yellow] plot[domain=-1:1,samples=40] ({\scale*(\x)^2},\scale*\x);
            \begin{scope}
                \clip (-\scale,-\scale) rectangle (\scale,\scale);
                \foreach \i in {1,...,4}{
                \pgfmathsetmacro{\k}{.1*\i};
                \pgfmathsetmacro{\coefa}{(1+sqrt(1-4*(\k)^2))/(2*\k)};
                \pgfmathsetmacro{\color}{10*(5+\i)};
                \pgfmathsetmacro{\dom}{1/sqrt(\coefa)}
                \draw[thick,color=yellow!\color!darkgreen] plot[domain=-\dom:\dom,samples=40] ({\scale*\coefa*(\x)^2},\scale*\x);
                }
            \end{scope}
            \draw[thick,color=yellow!50!darkgreen] (-\scale,0) -- (\scale,0);
            \draw[thick,color=yellow!50!darkgreen] (0,-\scale) -- (0,\scale);
            \begin{scope}
                \clip (-\scale,-\scale) rectangle (\scale,\scale);
                \pgfmathsetmacro{\k}{.3};
                \pgfmathsetmacro{\coefa}{(1+sqrt(1-4*(\k)^2))/(2*\k)};
                \pgfmathsetmacro{\dom}{1/sqrt(\coefa)}
                \draw[very thick,color=blue] plot[domain=0:\dom,samples=40] ({\scale*\coefa*(\x)^2},\scale*\x);
                \pgfmathsetmacro{\k}{4/17};
                \pgfmathsetmacro{\coefa}{(1+sqrt(1-4*(\k)^2))/(2*\k)};
                \pgfmathsetmacro{\dom}{1/sqrt(\coefa)}
                \draw[very thick,color=red] plot[domain={\dom*(240-\i)/240}:\dom,samples=40] ({\scale*\coefa*(\x)^2},\scale*\x);
            \end{scope}
        \end{scope}
        \node[below] at (4,4) {%
        \begin{minipage}[t]{7cm}
            For example, for $k = 3$, as $y$ approaches $0$, the point $(ky^2,y)$
            approaches the origin following the level curve corresponding to the
            level 
            \[\frac{3}{3^2 + 1} = 0.3\]
            For $k = 4$ the point follows a different level curve, and the limit
            will be \[\frac{4}{4^2+1} = \frac{4}{17}\]
        \end{minipage}%
        };
    \end{tikzpicture}%
    }%
\end{animateinline}%
\end{document}
