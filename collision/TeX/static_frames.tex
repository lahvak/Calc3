\documentclass[aspectratio=169]{beamer}
\let\digamma\relax
\usepackage[romanfamily=casual,lucidasmallscale, nofontinfo]{lucimatx}
%\usepackage[romanfamily=bright-osf, stdmathdigits=true,lucidasmallscale, nofontinfo]{lucimatx}
\linespread{1.04} %  this value suits  scale=0.9  or lucidasmallscale
%\usepackage{tikz}
%\usepackage{pgfplots}
%\pgfplotsset{compat=1.10}
\usepackage{siunitx}
\setbeamertemplate{navigation symbols}{}
\usefonttheme{professionalfonts,serif}
\newcolumntype{T}[1]{S[table-format=#1]}

\begin{document}
\begin{frame}
    First let's look at two lines with the following parametric equations:
    \vspace{1.5\baselineskip}
    \begin{columns}
        \begin{column}{0.5\textwidth}
            \[\begin{aligned}
                    x(t) &= 1 - .6t\\
                    y(t) &= -1 + t\\
                    z(t) &= -1 + .8t
                \end{aligned}\]
        \end{column}%
        \begin{column}{0.5\textwidth}
            \[\begin{aligned}
                    x(t) &= .2 - .2t\\
                    y(t) &= 1.6 - .3t\\
                    z(t) &= -.4 + .5t
                \end{aligned}\]
        \end{column}
    \end{columns}
    \pause
    \vspace{1.5\baselineskip}
    The following animation shows two moving objects with positions given by
    these equations for $t$ between 1 and 2.5.
\end{frame}
\begin{frame}
    We saw the two objects collide at $t=2$.\pause

    If we set the coordinates of the two objects equal to each other and solve
    for $t$, this is what happens:\pause

    \begin{columns}
        \begin{column}{.5\textwidth}
            The system
            \[
                \left\{
                    \begin{aligned}
                        1 - .6t &= .2 - .2t\\
                        -1 + t &= 1.6 - .3t\\
                        -1 + .8t &= -.4 + .5t
                    \end{aligned}
                \right.
            \]
        \end{column}\pause%
        \begin{column}{.5\textwidth}
            becomes
            \[
                \left\{
                    \begin{aligned}
                        -.4t &= -.8\\
                        1.3t &= 2.6\\
                        .3t &= .6
                    \end{aligned}
                \right.
            \]
        \end{column}
    \end{columns}\pause
    \vspace{1.5\baselineskip}
    which has a unique solution $t=2$.
\end{frame}
\begin{frame}
    Next let's look at these two lines:
    \vspace{1.5\baselineskip}
    \begin{columns}
        \begin{column}{0.5\textwidth}
            \[\begin{aligned}
                    x(t) &= 1.3 - .6t\\
                    y(t) &= -1.5 + t\\
                    z(t) &= -1.4 + .8t
                \end{aligned}\]
        \end{column}%
        \begin{column}{0.5\textwidth}
            \[\begin{aligned}
                    x(t) &= .2 - .2t\\
                    y(t) &= 1.6 - .3t\\
                    z(t) &= -.4 + .5t
                \end{aligned}\]
        \end{column}
    \end{columns}
\end{frame}
\begin{frame}
    If we set the coordinates of these two objects equal to each other and solve
    for $t$, this is what happens:\pause

    \begin{columns}
        \begin{column}{.5\textwidth}
            The system
            \[
                \left\{
                    \begin{aligned}
                        1.3 - .6t &= .2 - .2t\\
                        -1.5 + t &= 1.6 - .3t\\
                        -1.4 + .8t &= -.4 + .5t
                    \end{aligned}
                \right.
            \]
        \end{column}\pause%
        \begin{column}{.5\textwidth}
            becomes
            \[
                \left\{
                    \begin{aligned}
                        -.4t &= -1.1\\
                        1.3t &= 2.9\\
                        .3t &= 1
                    \end{aligned}
                \right.
            \]
        \end{column}
    \end{columns}\pause
    \vspace{1.5\baselineskip}
    This system has no solution.\pause

    The animation looks like this:
\end{frame}
\begin{frame}
    It seems that the two trajectories still pass through the same point, but
    at different times.  Let's try to solve this system instead:
    \vspace{1\baselineskip}

    \begin{columns}
        \begin{column}{.5\textwidth}
            \[
                \left\{
                    \begin{aligned}
                        1.3 - .6r &= .2 - .2s\\
                        -1.5 + r &= 1.6 - .3s\\
                        -1.4 + .8r &= -.4 + .5s
                    \end{aligned}
                \right.
            \]
        \end{column}\pause%
        \begin{column}{.5\textwidth}
            \[\text{or\ \ }
                \left\{
                    \begin{aligned}
                        -.6r + .2s &= -1.1\\
                        r + .3s &= 3.1\\
                        .8r - .5s &= 1
                    \end{aligned}
                \right.
            \]
        \end{column}
    \end{columns}\pause
    \vspace{1.5\baselineskip}
    The augmented matrix of the system is
    \[
        \left[
            \begin{array}{T{-1.1}T{-1.1}|T{-1.1}}
                -.6 & .2 & -1.1\\
                1 & .3 & 3.1\\
                .8 & -.5 & 1
            \end{array}
        \right]
    \]
\end{frame}
\begin{frame}
    \small
    \begin{columns}
        \begin{column}{.3\textwidth}
            Starting with:
            \[
                \left[
                    \begin{array}{T{-1.1}T{-1.1}|T{-1.1}}
                        -.6 & .2 & -1.1\\
                        1 & .3 & 3.1\\
                        .8 & -.5 & 1
                    \end{array}
                \right]
            \]\pause
            Switching the first and second row:
            \[
                \left[
                    \begin{array}{T{-1.1}T{-1.1}|T{-1.1}}
                        1 & .3 & 3.1\\
                        -.6 & .2 & -1.1\\
                        .8 & -.5 & 1
                    \end{array}
                \right]
            \]\pause
            Adding .6 times row 1 to row 2:
            \[
                \left[
                    \begin{array}{T{1.1}T{-1.2}|T{1.2}}
                        1 & .3 & 3.1\\
                        0 & .38 & .76\\
                        .8 & -.5 & 1
                    \end{array}
                \right]
            \]
        \end{column}\pause
        \begin{column}{.3\textwidth}
            Adding $-.8$ times row 1 to row 2:
            \[
                \left[
                    \begin{array}{T{1.0}T{-1.2}|T{-1.2}}
                        1 & .3 & 3.1\\
                        0 & .38 & .76\\
                        0 & -.74 & -1.48
                    \end{array}
                \right]
            \]\pause
            Dividing row 2 by $.38$:
            \[
                \left[
                    \begin{array}{T{1.0}T{-1.2}|T{-1.2}}
                        1 & .3 & 3.1\\
                        0 & 1 & 2\\
                        0 & -.74 & -1.48
                    \end{array}
                \right]
            \]\pause
            Dividing row 3 by $-.74$:
            \[
                \left[
                    \begin{array}{T{1.0}T{1.1}|T{1.1}}
                        1 & .3 & 3.1\\
                        0 & 1 & 2\\
                        0 & 1 & 2
                    \end{array}
                \right]
            \]
        \end{column}\pause
        \begin{column}{.3\textwidth}
            Subtracting row 2 from row 3:
            \[
                \left[
                    \begin{array}{T{1.0}T{1.1}|T{1.1}}
                        1 & .3 & 3.1\\
                        0 & 1 & 2\\
                        0 & 0 & 0
                    \end{array}
                \right]
            \]\pause
            Adding $-.3$ times row 2 to row 1:
            \[
                \left[
                    \begin{array}{T{1.0}T{1.0}|T{1.1}}
                        1 & 0 & 2.5\\
                        0 & 1 & 2\\
                        0 & 0 & 0
                    \end{array}
                \right]
            \]\pause

            The solution is
            \[
                \begin{aligned}
                    r &= 2.5\\
                    s &= 2
                \end{aligned}
            \]
        \end{column}\pause
    \end{columns}
\end{frame}
\begin{frame}
    The first object, at time $t = 2.5$, will be at the same place the second
    object passed through at time $t = 2$.\pause

    \vspace{3\baselineskip}

    The two lines will intersect, but the two objects will not collide.\pause
\end{frame}
\end{document}
