\documentclass[aspectratio=169]{beamer}
\let\digamma\relax
\usepackage[romanfamily=casual,lucidasmallscale, nofontinfo]{lucimatx}
%\usepackage[romanfamily=bright-osf, stdmathdigits=true,lucidasmallscale, nofontinfo]{lucimatx}
\linespread{1.04} %  this value suits  scale=0.9  or lucidasmallscale
\usepackage{tikz}
%\usepackage{pgfplots}
%\pgfplotsset{compat=1.10}
\setbeamertemplate{navigation symbols}{}
\usefonttheme{professionalfonts,serif}

\begin{document}
\begin{frame}
    The TNB frame consists of three vectors:\pause
    \begin{itemize}
        \item The \color{red}unit tangent vector\color{black}, that always points in
            the direction of the curve.\pause
        \item The \color{green}unit normal vector\color{black}, that is
            perpendicular to the tangent vector and always point in the
            direction the curve is turning.\pause
        \item The \color{blue}unit binormal vector\color{black}, that is
            perpendicular to both the tangent vector and the normal vector.
    \end{itemize}\pause
    Together they define a local rectangular coordinate system at each point on
    the curve.\pause

    First let's look at the TNB frame as it moves along the curve.
\end{frame}
\begin{frame}
    Now we will see the same animation again, but this time we will add the
    \emph{osculating circle} of the curve.  \pause
    
    Osculating circle is a circle that has the same TNB frame and the same
    \emph{curvature} as the curve. \pause

    It is the circle that provides the best approximation of the curve at the
    given point.
\end{frame}
\begin{frame}
    Finally, we will see the same animation again, however, this time we will
    take a seat at the roller coaster.\pause

    We will see the animation from a vantage point with coordinates $(-1.1,
    0.1, 0.1)$ \alert{in the TNB frame}, looking in the general direction of the center
    of the TNB frame.\pause

    You will see that the TNB frame will be seemingly stationary, while the
    curve will wind itself around.
\end{frame}
\end{document}
