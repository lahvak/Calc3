\documentclass[aspectratio=169]{beamer}
\let\digamma\relax
\usepackage[romanfamily=casual,lucidasmallscale, nofontinfo]{lucimatx}
%\usepackage[romanfamily=bright-osf, stdmathdigits=true,lucidasmallscale, nofontinfo]{lucimatx}
\linespread{1.04} %  this value suits  scale=0.9  or lucidasmallscale
\usepackage{bm}
\usepackage{tikz}
%\usepackage{pgfplots}
%\pgfplotsset{compat=1.10}
\setbeamertemplate{navigation symbols}{}
\usefonttheme{professionalfonts,serif}
\newcommand{\vect}[1]{\mbf{#1}}

\begin{document}
\begin{frame}
    In this video we will look at two common interpretation of \emph{vector
        functions} of a single real variable.\pause

    By a \emph{vector function} we will understand a function $\vect{r}:
    \mathbb{R}\to\mathbb{R}^3$, that is a function from the number line to the
    three dimensional space.\pause

    In $\mathbb{R}^3$ we will, as is a common practice, identify each point
    with its position vector, which will allow us to think of $\mathbb{R}^3$ as
    a space of points and a vector space, interchangeably. 
\end{frame}
\begin{frame}
    There is nothing special about $\mathbb{R}^3$, the same ideas would work
    with $\mathbb{R}^n$ for any positive integer $n$ (including 1), except that
    3 is the highest value of $n$ that makes this easy to visualize.\pause

    Think about how this would look like with $n=2$ and $n=1$ (in which case
    you just get regular real functions of single real variable).
\end{frame}
\begin{frame}{Trajectory of a moving object}
    Given a continuous function $\vect{r}:\mathbb{R}\to\mathbb{R}^3$, we can
    interpret the independent variable $t$ as \emph{time}, and the value
    $\vect{r}(t)$ as the position vector of a moving point in $\mathbb{R}^3$ at
    the time $t$.\pause

    As the point moves, it traces a trajectory, which is a curve in
    $\mathbb{R}^3$.\pause

    Each point on the ``time'' line corresponds to at most one point on the
    curve.  As the value of $t$ increases (time passes), the point
    $\vect{r}(t)$ traces the curve in $\mathbb{R}^3$.
\end{frame}
\end{document}
