\documentclass[aspectratio=169]{beamer}
\let\digamma\relax
\usepackage[romanfamily=casual,lucidasmallscale, nofontinfo]{lucimatx}
%\usepackage[romanfamily=bright-osf, stdmathdigits=true,lucidasmallscale, nofontinfo]{lucimatx}
\linespread{1.04} %  this value suits  scale=0.9  or lucidasmallscale
\usepackage{bm}
\usepackage{tikz}
%\usepackage{pgfplots}
%\pgfplotsset{compat=1.10}
\setbeamertemplate{navigation symbols}{}
\usefonttheme{professionalfonts,serif}
\newcommand{\vect}[1]{\mbf{#1}}

\begin{document}
\begin{frame}{Embedding into $\mathbb{R}^3$}
    Another way to look at a continuous vector function is a deformation and
    embedding of a (piece of) number line into the three dimensional
    space.\pause

    We can think about this as taking a number line (or a piece of a number
    line) and bending, compressing and stretching it while inserting it into
    $\mathbb{R}^3$.\pause

    Some parts of the line may get stretched out and some may get compressed
    during the process.\pause

    It may even happen that several points (or even infinitely many points)
    will end up at the same location in $\mathbb{R}^3$.\pause

    The location at which a point $t \in \mathbb{R}$ will end up is determined
    by the vector $\vect{r}(t)$.
\end{frame}
\end{document}
